\chapter{摘要}

神经元追踪与数字重建是神经科学界的热门方向。通过神经元追踪与数字重建,有助于推测大脑运作方式,理解智慧的产生。由于神经元拓扑结构十分复杂,自动化重建结果仍需要人工纠正和修改。现有的编辑软件大多运行在单机上,无法进行协同编辑。随着计算机性能与网络速度的提升,在线协同编辑成为了可能。

在这样的背景下,设计并实现一个在线多用户的神经元重建平台,利用互联网便于数据共享与交流的特点,帮助神经科学家便捷地进行异地协同编辑,并分享完成重建的结构脑图谱。将神经元结构抽象成树状结构,将操作抽象成点集变换,实现了神经元结构编辑操作的储存与传输。使用服务器缓存,客户端缓存,数据压缩,负载均衡等优化技术使平台足以满足一千名用户同时编辑的需求,并能在毫秒级别的时间内做出响应,达到了实时操作的要求,为神经科学家协同编辑神经元结构与分享成果提供了便利。

{
    \vspace{1em}
    \setlength{\parindent}{0em}
    \textbf{关键词} \; 生物图像信息, \; 神经元重建, \; 实时编辑平台 \par
}