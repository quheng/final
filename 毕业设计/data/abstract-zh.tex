\chapter{摘要}

根据原始神经元图像信息进行神经元追踪与数字重建是神经科学界热门方向。神经元的形态反应出它的功能,相同结构的神经元通常具有类似的功能。神经科学家通过神经元追踪与数字重建,可以反推大脑是如何运作的,对理解智慧的产生有重要的帮助。由于神经元拓扑结构的复杂性,一些自动化重建结果的细节上仍然需要研究人员对数字重建的结果进行人工纠正和修改,以确保数字重建工作的准确性,研究人员还需要对数字重建结果进行编辑,比如添加或删除一些网络分支等。另外现有的原始神经图像信息的神经元追踪和数字重建软件大多是运行在单机之上,无法满足多用户协同编辑与修改的需求,不利于结构脑图谱的交流。计算机性能和网络速度的提升使得在线实时编辑神经网络结构成为了可能。在这样的背景下,设计并实现了一个在线多用户的神经元网络结构编辑分享平台,利用互联网便于数据共享与交流的特点,使得神经科学家可以便捷地进行异地,多用户协同编辑神经元网络结构,并能分享完成重建的结构脑图谱,探索神经元结构下的奥秘。项目中使用了 DVID 数据库储存神经元信息,PostgreSQL 储存用户信息并使用 Node.js 与 Express 完成了网络服务器的搭建,为前端可视化操作提供了有力的支持。实验与测试结果表明,该平台足以支撑至少数千名用户同时编辑的需求,并能在毫秒级别的时间内给出做出相应,达到了实时操作的要求,为神经科学家在神经元追踪以及数字重建方面提供了便利。

{
    \vspace{1em}
    \setlength{\parindent}{0em}
    \textbf{关键词} \; 生物图像信息 \; 神经元重建 \; 实时编辑平台 \par
}
