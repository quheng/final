%% 英文摘要
\chapter*{\centerline{Abstract}}
\chaptermark{Abstract}
\addcontentsline{toc}{chapter}{Abstract}

\vspace{1em}
Neuron tracing and digital reconstruction from original neural image information is a hot direction of neural science. The morphology of neurons reflects its function, the neurons with same morphology usually have similar function. By the reconstruction of brain connectome, neuroscientists can speculate how the brain works which is helpful to understand the intelligence. Because of the complexity of the topological structure of neurons, and in some details on the results of automation reconstruction researchers still need to manually correct and modify the results of digital reconstruction, in order to ensure the accuracy of the digital reconstruction. In addition, researchers need to edit the result of the digital reconstruction such as add or delete some branches, etc. Existing neurons tracking for original neural image and digital reconstruction softwares are mostly run on stand-alone, unable to meet the requirements of multi-user collaborative editing and modification, but also not conducive to the exchange of brain connectom. Improvements in computer performance and network speed make it possible for online real-time editing of neural network structures. In such a background, We have designed and implemented a neural network structure of online multi-user edit sharing platform, using the Internet which can facilitate data sharing and exchange can easily carry out off-site, edit neural network structure with multi-user cooperation, and share brain connectome, explore the mystery of neuronal structure. Project uses DVID database to store neural information, PostgreSQL to store user information and Node.Js, Express to build web server which provides the powerful support for the front-end visualization. 
Experiments and test reports show that the platform is sufficient to support the needs of at least thousands of users editing at the same time, response in milliseconds, meets the real-time operation requirements which provides convenience for neurologists in neuronal tracking and digital reconstruction.

\vspace{1em}

\noindent\textbf{Keywords}: bioimage informatics; neuron tracing; DVID; real-time editing platform
