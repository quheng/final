\chapter{整体架构与技术选型}

\section{整体架构}
项目整体结构如图 \ref{server} 所示,共分为神经信息数据库,用户信息数据库以及网络服务器三部分组成。
\begin{figure}
\centering
\includegraphics[width=148mm]{images/server}
\caption{项目整体架构}
\label{server}
\end{figure}


\section{技术选型}

\subsection{神经信息数据库}
神经信息数据库包含两部分数据,一部分是原始大脑切片显微镜图像,另外一部分是初步完成数字重建的 SWC 文件。使用 DVID 作为数据库,储存这些信息。DVID 是一个分布式面向图像的数据服务,主要用于图像分析与可视化。DVID 有如下特点:

1. 便于扩展数据类型,允许用户根据数据特点加速访问速度,减少储存空间,提供方便的 API。这为储存数字重建结果提供了便利。

2. 为分布式数据储存提供了类似于 GIT 的版本控制系统,在此基础之上我们可以解决多用户同时编辑产生冲突的问题。

3. 方便连接其他 API 如 Google BrainMaps 和 OpenConnectome 等。

4. 支持多分辨率图像数据,使得用户可以在不同尺度下观察图像信息。

在 DVID 的基础上,构建出原始大脑切片显微镜图像与数字重建结果的储存仓库,将数据储存抽象成数据存储服务,使得可以专注于完成核心算法和逻辑。

\subsection{用户信息数据库}
用户信息数据库包含多用户管理以及用户资源管理。使用 PostgreSQL 数据库储存这部分信息。PostgreSQL 最初由加州大学伯克利分校计算机系开发完成。在支持大部分 SQL 标准之上,提供了许多诸如复杂查询,多版本并行控制,事物完整性等现代特性\upcite{stonebraker1991postgres}。由于 PostreSQL 对标准 SQL 支持度较高,可以方便的和 DVID 联系起来,将用户信息和原始图像信息,数字重建结果对应起来。利用 PostgreSQL 支持的储存过程,事物以及多版本并行控制特性,我们可以方便的实现分布式,多用户实时编辑平台,并解决多用户同时编辑可能产生冲突的问题。

\subsection{网络服务器}
采用 Node.js 和 Express 完成网络应用开发。Node.js 是一个基于 Chrome V8 引起的 JavaScript 的运行环境。Node.js 使用了一个事件驱动、非阻塞式 I/O 的模型,使其轻量又高效\upcite{tilkov2010node}。Express 是一个基于 Node.js 平台的极简、灵活的 web 应用开发框架,提供丰富的 HTTP 快捷方式和任意排列组合的 Connect 中间件,帮助快速、简单的创建健壮、友好的 API。