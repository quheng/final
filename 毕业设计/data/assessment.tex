{
  \setlength{\parindent}{0em}
  \linespread{1}

  \vspace*{-2.1em}

  {
    \centering
    \songti\xiaoer\bfseries
    毕~~业~~论~~文~~(设~~计)~~考~~核 \par
  }

  \vspace{1.1em}

  {
    \songti\sihao\bfseries
    一、\; 指导教师对毕业论文(设计)的评语 \par
  }
    ~~~~~~针对大规模神经元重建任务的实际需求,论文设计并实现了支持1000个用户操作的服务器版本,进行了系统级的优化,并进行了服务器运行参数的广泛测试。进一步的工作将在硬件系统到位后,由用户进行实际操作测试,以推进系统的进一步真正的实用化过程。毕业设计推进顺利,完成了预定的目标。

  \vspace{8em}

  {
    \songti\xiaosi\bfseries
    \hfill 指导教师(签名) \; \underline{\hspace{5em}}

    \vspace{0.1em}

    \hfill \hspace{2em} 年 \hspace{1em} 月 \hspace{1em} 日 \par
  }

  \vspace{0.7em}

  {
    \songti\sihao\bfseries
    二、 \; 答辩小组对毕业论文(设计)的答辩评语及总评成绩:
  }

  \vspace{14.7em}

  {
    \renewcommand{\arraystretch}{1.5}
    \songti\xiaosi\bfseries
    \hfill \begin{tabular}{|c|m{4.1em}|m{4.1em}|m{4.1em}|m{9.1em}|c|}
      \hline
      成绩比例 & {\centering 开题报告 \\ 占(20\%)} & {\centering 外文翻译 \\ 占(10\%)} & {\centering 中期报告 \\ 占(10\%) } & {\centering 毕业论文(设计) \\ 质量及答辩占(60\%)} & 总成绩 \\
      \hline
      分值 & & & & & \\
      \hline
    \end{tabular} \par
  }

  \vspace{2em}

  {
    \songti\xiaosi\bfseries
    \hfill 答辩小组负责人(签名) \; \underline{\hspace{5em}}

    \vspace{0.1em}

    \hfill \hspace{2em} 年 \hspace{1em} 月 \hspace{1em} 日 \par
  }
}
