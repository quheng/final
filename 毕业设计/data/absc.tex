%% 中文摘要
\chapter*{\centerline{摘\quad 要}}
\chaptermark{摘要}
\addcontentsline{toc}{chapter}{摘要}

\vspace{1em}
原始神经图像信息的神经元追踪和数字重建是神经科学界热门方向。神经元的形态反应出它的功能,相同功能的神经元通常具有类似的功能。神经科学家通过结构脑图谱的重建,可以反推大脑是如何运作,对理解智慧的产生有重要的帮助。由于神经元拓扑结构的复杂性,在一些自动化重建结果的细节上仍然需要研究人员对数字重建的结果进行人工纠正和修改,以确保数字重建工作的准确性。另外研究人员需要对数字重建结果进行编辑,比如添加或删除一些网络分支等。为了便于研究人员编辑数字重建的结果,根据“所见即所得”的原则,设计出了 SWC 格式。现有的用于原始神经图像信息的神经元追踪和数字重建的软件大多是运行在单机上,无法满足多用户协同编辑与修改的需求,也不利于结构脑图谱的交流。计算机性能和网络速度的提升使得在线实时编辑神经网络结构成为了可能。在这样的背景下,设计并实现一个了在线多用户的神经元网络结构编辑分享平台,利用互联网便于数据共享的特点,帮助神经学研究人员便捷地进行异地,多用户协同编辑神经元网络结构,并能分享结构脑图谱,共同探索神经元结构下的奥秘。本文论述了在线交互式神经元重建系统的服务器架构及其实现方式,

\vspace{1em}

\noindent\textbf{关键词}:\quad 此处填入论文关键词
