\chapter{总结与展望}
本文设计并实现了一个在线多用户的神经元网络结构编辑分享平台,利用互联网便于数据共享的特点,帮助神经学研究人员便捷地进行异地,多用户协同编辑神经元网络结构,并能分享结构脑图谱,共同探索神经元结构下的奥秘。完成平台搭建后,对平台进行了详细的压力测试,并根据压力测试的结果进行性能调优使之可以支持数千名用户的实时编辑操作。

限于时间关系,平台完成部署后并没有真正投入使用,但是已经与一些神经科学家取得了联系,即将投入神经科学有关结构脑胞体重建和编辑的教学任务中。限于实验室硬件环境尚未完全部署完毕,只有三台计算机可以使用,不能充分发挥分布式结构的优势,在实验室硬件部署完毕后,可以将平台部署在更多的计算机上,充分发挥分布式架构可扩展的特性使之可以支撑更多的用户同时在线。同时结构脑胞体自动重建算法仍然使用的是单机版,脑胞体自动重建算法的并行化已经用 Spark 实现,即将可以在平台中使用,这将进一步提升平台的并行化程度。