\chapter{总结与展望}

\section{项目总结}
本文设计并实现了一个在线多用户的神经元网络结构编辑分享平台,利用互联网便于数据共享的特点,帮助神经学研究人员便捷地进行异地,多用户协同编辑神经元网络结构,并能分享结构脑图谱,共同探索神经元结构下的奥秘。将用户对神经元结构的编辑操作抽象成不同层次的操作序列,使之可以序列化到数据库之中并支持多用户操作合并,操作撤销等较为复杂的功能。

项目完成部署后,通过一系列的性能优化,将平台的响应时间控制在数十毫秒的级别,达到了实时编辑操作的需求。同时平台每秒钟可以处理一千次用户的请求,足以支持一千名用户同时在线,满足了团队合作与协同编辑的需求,为神经科学家进行在线交互式神经元编辑与分享提供了便利。

\section{存在的问题与展望}
最主要的问题是平台完成部署后并没有真正投入使用,但是已经与一些神经科学家取得了联系,即将投入神经科学有关结构脑胞体重建,编辑与分享的教学任务中。另外压缩过后的 js 文件依旧有 4MB 左右的大小,用户在第一次访问时能够明显的感受到加载时间,希望后续能够进一步压缩文件大小,减少加载时间。限于实验室硬件环境尚未完全部署完毕,目前只有三台计算机可以使用,不能充分发挥分布式架构的优势。在实验室硬件部署完毕后,可以将平台部署在更多的计算机上,充分发挥分布式架构可扩展的特性使之可以支撑更多的用户同时在线。同时结构脑胞体自动重建算法仍然使用的是单机版,脑胞体自动重建算法的并行化已经用 Spark 实现,即将可以在平台中使用,这将进一步提升平台的并行化程度。