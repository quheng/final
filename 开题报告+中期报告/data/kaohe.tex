\thispagestyle{empty}

{
  \setlength{\parindent}{0em}
  \renewcommand{\baselinestretch}{2}

  {
    \stfangsong\sanhao\bfseries
    \centering
    毕业设计开题报告、外文翻译的考核 \par
  }

  {
    \songti\sihao\bfseries
    导师对开题报告、外文翻译的评语及成绩评定:

    {
      \setlength{\parindent}{2em}
      \songti\sihao\mdseries\indent
        开题认真分析了昆虫脑部神经回路重建的服务器设计需求,进行了详细的系统设计,前期认真进行了编码,为后续工作的开展,奠定了良好的基础。外文翻译认真,规范。
    }
    \vspace{3em}

    {
      \renewcommand{\baselinestretch}{1}

      \begin{flushright}

        \begin{tabular}{|c|c|c|c|}
          \hline
          成绩比例 & \parbox[c]{3.6em}{\xiaosi 开题报告 \\ 占(20\%) \vspace{0.25em}} & \parbox[c]{3.6em}{\xiaosi 外文翻译 \\ 占(10\%) \vspace{0.25em}} \\
          \hline
          分值 & 18 & 9 \\
          \hline
        \end{tabular}

        \vspace{2em}

        {
          \songti\xiaosi\bfseries
          导师签名 \; \underline{\hspace{6em}} \\
          年 \qquad 月 \qquad 日 \par
        }
      \end{flushright}
    }
  }

  \vspace{3em}

  {
    \songti\sihao\bfseries
    学院盲审专家对开题报告、外文翻译的评语及成绩评定:

    {
      \setlength{\parindent}{2em}
      \songti\sihao\mdseries\indent
      开题报告内容完整,准确表达了毕业论文的目标和任务,对现有相关工作和技术已做了分析,为项目的执行奠定了基础。技术路线可行,时间进度安排较合理。达到开题报告要求。 文献翻译准确,流畅,工作量饱满。
    }
    \vspace{3em}

    {
      \renewcommand{\baselinestretch}{1}

      \begin{flushright}

        \begin{tabular}{|c|c|c|c|}
          \hline
          成绩比例 & \parbox[c]{3.6em}{\xiaosi 开题报告 \\ 占(20\%) \vspace{0.25em}} & \parbox[c]{3.6em}{\xiaosi 外文翻译 \\ 占(10\%) \vspace{0.25em}} \\
          \hline
          分值 & 18 & 9 \\
          \hline
        \end{tabular}

        \vspace{2em}

        {
          \songti\xiaosi\bfseries
          开题报告审核负责人(签名/签章) \; \underline{\hspace{6em}} \par
        }
      \end{flushright}
    }
  }

  \newpage

  {
    \stfangsong\sanhao\bfseries
    \centering
    毕业设计中期报告考核 \par
  }

  {
    \songti\xiaosi\bfseries
    导师对中期报告的评语及成绩评定:

    {
      \setlength{\parindent}{2em}
      \songti\sihao\mdseries\indent    完成了多用户SWC编辑功能的后台服务器的基本功能实现,包括数据库、后台并发支持等;采用Gatling进行了关键参数评测。毕业设计整体进展顺利。
    }
    \vspace{10em}

    {
      \renewcommand{\baselinestretch}{1}

      \begin{flushright}

        \begin{tabular}{|c|c|c|c|}
          \hline
          成绩比例 & \parbox[c]{3.6em}{\xiaosi 中期报告 \\ 占(10\%) \vspace{0.25em}} \\
          \hline
          分值 & 9 \\
          \hline
        \end{tabular}

        \vspace{2em}

        {
          \songti\xiaosi\bfseries
          导师签名 \; \underline{\hspace{6em}} \\
          年 \qquad 月 \qquad 日 \par
        }
      \end{flushright}
    }
  }
}
